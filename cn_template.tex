% !TeX encoding = UTF-8
% @Yongjian.Li 澳門科技大學-畢業論文 LaTeX template-2023.6.10
% 使用手冊: https://iihciyekub.github.io/must-thesis-manual

\documentclass[writingLanguage=chinese,
    addPageTitle=on,
    addDeclaration=on,
    addMUSTlog=off,
    addFigTOC=on,   
    addTabTOC=on,
    refIndent=on,
    printMod=off,
]{.def/must}
   
% 論文基本信息,必填不能刪除
\def\shool              {Macau University of Science and Technology}
\def\cnTitle            {XXX 銀行(澳門分行)與 XXX 銀行合併之研究}
\def\cnShortTitle       {\cnTitle}% 頁眉顯示的中文論文短題目
\def\enTitle            { }
\def\enShortTitle       {\enTitle}% 頁眉顯示的英文論文短題目
\def\Name               {模板}% 名稱
\def\StudentNo          {1809853G-BM30-0053}% 學號
\def\Faculty 	        {商學院}% 所在學院
\def\Program 	        {管理學博士學位}% 學位名稱
\def\Major              {商業量化}% 專業名稱
\def\Supervisor	        {李新 副教授}% 指導老師
\def\DateofWriting		{\datea\today}% 設置論文寫作完成時間
\def\DateofDeclaration	{\dateb\today}% 設置論文原創聲明時間
\def\DateofSignature	{2023/06/30}% 設置簽署論文原創聲明的時間
\def\PublicAfterYears   {5}% 設置論文幾年後公開

\usepackage{pgf}




\begin{document}

\begin{abstract@cn}{關鍵字1、關鍵字1、關鍵字1、關鍵字1}
% 中文摘要

\end{abstract@cn}

\begin{abstract@en}{keyword1、keyword1、keyword1、keyword1、}
% 英文摘要

\end{abstract@en}

%%%%%%%%% 添加目錄 
\addtableofcontents


%%%%%%%%% 正文部分
\chapter{第一章}
\section{...}
\subsection{...}
...


%%%%%%%%% 参考文献页面
% bib文件名必须为ref.bib; 請利用转换工具,将bib 转为 bbl 格式,文件名为 ref.bbl
\addreference


%%%%%%%%% 附录页面
\begin{appendix}
證明過程
\end{appendix}



%%%%%%%%% 致谢页面
\begin{acknowpage}
謝謝各位 

省略
\end{acknowpage}




% 填寫個人簡歷
\begin{addcvpage}
% 設置入學時間
\addedudate{2019 年 7 月}

% 填寫教育經歷,注意內容以逗號作分隔,
\addeduItem{2009.16-2012.13,中國澳門氹仔島澳門科技大學,商學院}
\addeduItem{2009.16-2012.13,澳門科技大學,商學院}
\addeduItem{2009.16-2012.13,澳門科技大學,商學院}


%增加學術文章
\addpaperItem{ 
    \item 中國澳門氹仔島澳門科技大學1中國澳門氹仔島澳門科技大學1中國澳門氹仔島澳門科技大學1中國澳門氹仔島澳門科技大學1中國澳門氹仔島澳門科技大學1中國澳門氹仔島澳門科技大學1中國澳門氹仔島澳門科技大學1中國澳門氹仔島澳門科技大學,
    \item 中國澳門氹仔島澳門科技大學,
}


% 增加項目
\addprojectItem{
    \item 中國澳門氹仔島澳門科技大學,商學院商學院
    \item  商學院商學院商學院中國澳門氹仔島澳門科技大學,商學院商學院商學院商學院商學院中國澳門氹仔島澳門科技大學,商學院商學院商學院商學院商學院
}
\end{addcvpage}






\end{document}