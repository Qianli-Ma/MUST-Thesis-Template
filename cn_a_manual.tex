% !TeX encoding = UTF-8
% @Yongjian.Li 澳門科技大學-畢業論文 LaTeX template-2024.04.21
% 使用手冊: https://iihciyekub.github.io/must-thesis-manual

% 選擇加載項
\documentclass[writingLanguage=chinese,
    addPageTitle=on,
    addDeclaration=on,
    addMUSTlog=off,
    addFigTOC=on,   
    addTabTOC=on,
    refIndent=off,
    printMod=off,
]{.def/must}
   
% 論文基本信息,必填不能刪除
\def\shool              {Macau University of Science and Technology}
\def\cnTitle            {我的畢業論文中文題目}
\def\cnShortTitle       {\cnTitle}% 頁眉顯示的中文論文短題目
\def\enTitle            {English title of my graduation thesis}
\def\enShortTitle       {\enTitle}% 頁眉顯示的英文論文短題目
\def\Name               {李永建}% 名稱
\def\StudentNo          {1809853G-BM30-0053}% 學號
\def\Faculty 	        {商學院}% 所在學院
\def\Program 	        {管理學博士學位}% 學位名稱
\def\Major              {商業量化}% 專業名稱
\def\Supervisor	        {李新 副教授}% 指導老師
\def\DateofWriting		{\datea\today}% 設置論文寫作完成時間
\def\DateofDeclaration	{2023/11/20}% 設置論文原創聲明時間
\def\DateofSignature	{2023/11/30}% 設置簽署論文原創聲明的時間
\def\PublicAfterYears   {5}% 設置論文幾年後公開
\def\mySign             {Fig/mySign.png}%支持帶透明通道的png照片(推薦)
\def\supervisorSign     {Fig/supervisorSign.png}%支持帶透明通道的png照片





\begin{document}
\begin{abstract@cn}{澳門科技大學研究生畢業論文模板}
本模板基於\LaTeX{}文檔語言製作,嚴格符合澳門科技大學(以下簡稱“MUST”)的排版參考標準,涵蓋扉頁、原創性聲明、中英文摘要、目錄、圖表目錄、校徽水印、個人簡歷及參考文獻等內容。

\noindent 本排版設計嚴格遵守MUST的官方規範文件,包括:
\begin{enumerate}[label=\arabic*).]
\item 研究生論文寫作指導(2022年10月修訂版)
\item 碩士與博士論文參考文獻補充説明\_2022(商學院)
\item 扉頁格式要求
\item 學位論文原創性聲明要求
\item 校徽水印規範
\end{enumerate}
請注意,除商學院外的其他學院可能需要補充或修改部分排版細節。所有模板文件均存於.def文件夾內,且已添加詳盡的中文註釋,用户可根據具體排版要求進行調整。如需技術支持,請訪問\faGithub:\url{https://github.com/iihciyekub/MUST-Thesis/issues} 提交問題以獲得進一步的幫助。

\noindent 最後更新時間:\datea\today

\end{abstract@cn}

\begin{abstract@en}{Macau University of Science and Technology Graduate Thesis Template}
This template is created based on the \LaTeX{} documentation language and strictly complies with the typesetting standards provided by Macau University of Science and Technology (hereinafter referred to as 'MUST'), covering the title page, declaration of originality, abstracts in both Chinese and English, table of contents, list of figures, university emblem watermark, personal resume, and references.

\noindent The typesetting design strictly follows MUST's official specification documents, including:
\begin{enumerate}[label=\arabic*).]
\item Graduate Thesis Writing Guide (Revised October 2022)
\item Supplemental Instructions for References in Master's and Doctoral Theses 2022 (Business School)
\item Title Page Format Requirements
\item Degree Thesis Originality Declaration Requirements
\item University Emblem Watermark Standards
\end{enumerate}
Please note that faculties other than the Business School may need to supplement or modify some typesetting details. All template files are stored in the .def folder, with extensive Chinese comments added, allowing users to adjust according to specific typesetting requirements. For technical support, please visit \faGithub: \url{https://github.com/iihciyekub/MUST-Thesis/issues} to submit issues for further assistance.

\noindent Last updated: \datea\today

\end{abstract@en}

% 添加目錄 
\addtableofcontents

\chapter{模板設計開源}
\noindent \faHandORight\; 推薦使用 overleaf 在線版本 (持續更新) 
\begin{itemize}
    \item \url{https://www.overleaf.com/read/mjzpcxztzqzv#3b0b73}
\end{itemize}

\noindent \faHandORight\;  GitHub 源文件 
\begin{itemize}
    \item \url{https://github.com/iihciyekub/MUST-Thesis}
\end{itemize}

\noindent \faHandORight\; overleaf texAide (chrome Extensions) 

提供簡繁體轉換/澳科大參考文獻格式轉換/ \LaTeX{} 宏命令查詢 
\begin{itemize}
    \item \url{https://chromewebstore.google.com/detail/overleaf-texaide/icekiliecbhnockmfkehoebbkmhmapmo?hl=zh-CN}
\end{itemize}

\noindent \faHandORight\; 設計文檔更新歷史 
\begin{itemize}
    \item \url{https://github.com/iihciyekub/MUST-Thesis/blob/master/readme.md}
\end{itemize}

\noindent \faHandORight\; 在線幫助文檔 
\begin{itemize}
    \item \url{https://iihciyekub.github.io/must-thesis-manual/}
\end{itemize}





\chapter{論文排版説明}
\section{論文排版基本要求}
\subsection{排版要求規範文件}
\begin{table}[H]
\Large
\centering
\caption{MUST 校方排版要求規範文件,2024}
\begin{tabularx}{\textwidth}{lX}
\toprule
序 & 學校排版要求文件 (點擊查看) \\
\midrule
1& \faHandORight\; \href{https://www.must.edu.mo/images/GSO/files/sgsdocument/GS002.pdf}{研究生論文寫作指導 (2022年10月修訂)}\\
2& \faHandORight\; \href{https://www.must.edu.mo/images/GSO/files/sgsdocument/GS004.pdf}{扉頁格式}\\
3& \faHandORight\; \href{https://www.must.edu.mo/images/GSO/files/S023學位論文原創性聲明BI.pdf}{學位論文原創性聲明}\\
4& \faHandORight\; \href{http://www.must.edu.mo/images/SGS/files/APA_7th_0710.pdf}{研究生論文格式參考資料 (APA)}\\
5& \faHandORight\; \href{https://www.must.edu.mo/images/MSB/files/碩士與博士論文參考文獻格式補充説明_2022.pdf}{碩士與博士論文參考文獻格式補充説明\_2022 (商學院)}\\
6& \faHandORight\; \href{https://lib.must.edu.mo/sites/default/files/must-logo.jpg}{校徽水印}\\
\bottomrule
\end{tabularx}
\end{table}

\subsection{版面設定}
\begin{table}[H]
    \Large
    \centering
    \caption{讀取csv生成的表格}
    \begin{tabular}{
        >{\raggedright\arraybackslash}p{2.1cm}
        >{\raggedright\arraybackslash}p{12cm}
    }  % 第二列寬度調整為12cm
    \toprule
    % 表頭(根據你的CSV文件內容自定義)
    域 & 尺寸 \\ \midrule
    % 讀取CSV文件內容
    \csvreader[head to column names]{Tab/reftab.csv}{}
    { \csvcoli & \csvcolii \\ }\\[-1.5em]
    \bottomrule
    \end{tabular}
\end{table}

\subsection{論文排列順序}
扉頁、中文摘要、英文摘要、目錄、緒論、論證章、結論和建議、參考文獻、附錄
(圖表或文字資料) 、致謝、個人簡歷、及本人發表的論文和著作 (Publication)。






 

\ifthenelse{\equal{\Name}{模板}}{\addinfo}{}

\chapter{緒論}
\section{數學分析相關概念}

\begin{axiom}[皮亞諾公理 1]
maseeselect
\begin{enumerate}[label=\Alph*.]
\item 零是一個自然數:$0$ 是一個自然數。
\item 每個自然數都有一個後繼:對於每個自然數 $n$,存在一個唯一的自然數 $n+1$,稱為 $n$ 的後繼。
\item 不同的自然數具有不同的後繼:如果 $n$ 和 $m$ 是自然數,並且 $n+1 = m+1$,那麼 $n = m$。
\item 零不是任何自然數的後繼:對於任何自然數 $n$,$n+1$ 不等於 $0$。
\item 歸納原理(數學歸納法):如果一個集合 $S$ 滿足以下兩個條件,即 (a) $0$ 屬於 $S$,並且 (b) 對於每個自然數 $n$,如果 $n$ 屬於 $S$,則 $n+1$ 也屬於 $S$。那麼,$S$ 包含了所有的自然數。
\end{enumerate}
\end{axiom}

\begin{theorem}[費馬定理]
對於任意大於2的整數$n$,方程 $a^n + b^n = c^n$ 沒有整數解。
\end{theorem}



\begin{definition}
    一個素數是隻能被1和自身整除的正整數。
\end{definition}
\begin{example}
    設 $a$ 和 $b$ 是實數,那麼 $a+b=b+a$。
\end{example}

\begin{property}
    任意實數 $x$ 的平方 $x^2$ 非負。
\end{property}
\begin{proposition}
    對於任意實數 $x$,如果 $x > 0$,則 $x^2 > 0$。
\end{proposition}
\begin{lemma}
這是第二個引理 lemma。
\end{lemma}

\begin{corollary}
這是第二個推論 corollary。
\end{corollary}

\begin{remark}
這是註解 remark。
\end{remark}



\begin{condition}
這是條件 condition。
\end{condition}

 \begin{quote}
     這是一段引用的段落
 \end{quote}

\begin{assumption}
這是假設 assumption。
\end{assumption}



\begin{sidewaystable}[!htp]
    % 設置表格的列間距
    \setlength{\tabcolsep}{10mm}
    \centering
    \caption{旋轉表標題}
    % 下面一行代碼創建了一個左對齊的表格,並且具有 3 列。lcc 表示第一列左對齊,其餘 2 列都是居中對齊
    \begin{tabular}[l]{lcc}
    \toprule
        % 這裏填寫表頭
        col1 & col2 & col3 \\
    \midrule
        % 這裏填寫表格內容
        value1 & value2 & value3 \\
    \bottomrule
    \end{tabular}
\end{sidewaystable}


% \captionsetup[figure]{singlelinecheck=off,justification=raggedright}
% \captionsetup[subfigure]{singlelinecheck=on}
% \tikzstyle{every pin}=[fill=white,draw=black,font=\small,]
\begin{figure}[H]
	\centering
	\begin{subfigure}{0.49\textwidth}
	  	\centering
            \includegraphics[height=4cm]{example-image}
		\caption{A subfigure}
		\label{fig:sub1}
	\end{subfigure}
	\hfill
	\begin{subfigure}{.49\textwidth}
		\centering
            \includegraphics[height=4cm]{example-image-a}
		\caption{A subfigure}
	  	\label{fig:sub2}
	\end{subfigure}
        \newline
	\begin{subfigure}{.49\textwidth}
		\centering
            \includegraphics[height=4cm]{example-image-plain}
		\caption{A subfigure}
		\label{fig:sub3}
	\end{subfigure}
	\begin{subfigure}{.49\textwidth}
		\centering
            \includegraphics[height=4cm]{example-image-empty}
		\caption{A subfigure}
		\label{fig:sub4}
	\end{subfigure}
	\caption{A figure with two subfigures}
	\label{fig:sub}
\end{figure}
上面示例中,子圖\ref{fig:sub1}、子圖\ref{fig:sub2}、子圖\ref{fig:sub3}、子圖\ref{fig:sub4}分別表示子圖



\section{表}

\begin{table}
\caption{這是表的標題}
\centering
\begin{tabularx}{\textwidth}{XXX} % 表格寬度為網頁的一半
\toprule
內容1 & 內容2 & 內容3 \\
\midrule
行1 & 行1 & 行1\footnote{這是表格腳註的內容。} \\
行2 & 行2 & 行2 \\
\bottomrule
\end{tabularx}
\caption*{這是表格腳註}
\end{table}





\subsection{研究目的}

\txtHere{[1]}

\subsection{研究方法}

本文采用了隨機控制實驗設計,結合問卷調查和實驗數據分析的方法,對研究對象進行了深入的研究。

\subsection{主要貢獻}

\txtHere{[1]}

\section{相關工作}

本章將介紹相關領域的研究現狀,總結前人的工作成果,分析其不足之處,為後續研究提供理論和實踐基礎。

\subsection{前人工作總結}

\txtHere{[1]}

\subsection{前人工作不足}

\txtHere{[1]}

\section{研究內容}

本章將詳細介紹本文的研究內容和研究結果,包括實驗設計、實驗過程、實驗結果分析等。

\subsection{實驗設計}

\txtHere{[1]}

\subsection{實驗過程}
下面是一些wasysym宏包提供的符號示例:

二進制運算符:$\oplus$、$\ominus$、$\otimes$等。

邏輯運算符:$\wedge$、$\vee$、$\neg$等。

集合運算符:$\cup$、$\cap$、$\subset$、$\supset$等。

幾何圖形:$\square$、$\triangle$、$\circ$等。

天文符號:$\sun$、$\mercury$、$\venus$、$\earth$等。

卡牌符號:$\heartsuit$、$\diamondsuit$、$\clubsuit$、$\spadesuit$等

\subsection{實驗結果分析}

\txtHere{[1]}


 
\subsection{研究總結}

\txtHere{[1]}

\subsection{未來研究方向}

\txtHere{[1]}



% 請確保bib 文件名稱為 ref.bib, 利用js文件處理後的bbl文件名稱為 ref.bbl
\addreference


\begin{appendix}
證明過程
\end{appendix}


\begin{acknowpage}
謝謝各位 
省略
\end{acknowpage}



% 填寫個人簡歷
\begin{addcvpage}
% 設置入學時間
\addedudate{2019 年 7 月}

% 填寫教育經歷,注意內容以逗號作分隔,
\addeduItem{2009.16-2012.13,澳門科技大學,商學院}
\addeduItem{2009.16-2012.13,澳門科技大學,商學院}
\addeduItem{2009.16-2012.13,澳門科技大學,商學院}


%增加學術文章
\addpaperItem{ 
    \item 中國澳門氹仔島澳門科技大學 
    \item 中國澳門氹仔島澳門科技大學,
}

% 增加項目
\addprojectItem{
    \item 無
}
\end{addcvpage}


\end{document}