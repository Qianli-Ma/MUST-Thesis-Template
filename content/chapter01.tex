\section{示例:一級標題}
\subsection{示例:二級標題}
緒論的主要作用是:要告訴讀者本文的研究主題、論證本研究主題的價值所在、提出作者對研究問題的主觀答案。

\par 在此處的內容主要包括:
\begin{enumerate}
	\item[] (1) 問題實際背景和問題界定(明確實用價值及對實際問題的專業術語表述);
	\item[] (2) 文獻綜述理(論價值定位);
	\item[]	(3) 尚待解決問題(參照點選擇);
	\item[] (4) 假設提出(創新點提出);
\end{enumerate}


\subsection{示例:二級標題}
\par 通俗地講就是:研究動機、問題背景,選題原因和實際工作的關係、研究的
重要性、研究目的、研究假設或待解決問題、名詞及定義以及研究範
圍和限制等。

\subsubsection{示例:三級標題}
\par  問題實際背景和問題界定:選題的途徑:
(1)從閱讀文獻著手;
(2)從觀察現象著手。問題實際背景是用事實和現象來描述研究問題所在和其重要性。問題界定是指用專業述語來表述所要研究的問題。
\clearpage

\subsubsection{示例:三級標題}
\par 文獻綜述:是描述目前的研究現狀並作簡要分析。可以反映作者研究
的功力和閱讀文獻的數量,是否找到研究問題的關鍵文獻及抓准文獻
的重點。評述是否切中要害,是否有獨到見解。忌諱採用講義式將有
關研究課題的理論和學派簡要地陳述一篇;忌諱輕率批評前人的不足
和錯誤;忌諱含糊不清,採用的觀點和內容不清楚來源。綜述所引用
的文獻應主要選自學術期刊或學術會議的文章。教科書或其他書籍只
能占小部分。報章雜誌的觀點不能作為論證的依據。
尚待研究問題和假設的提出:以“尚待研究問題”來指出目前研究的不足
之處,然後提出研究的假設。它表述論文的創新點所在。它是對實際
問題觀察思考和閱覽前人研究工作的結果,又是論文隨後論證工作的
起點和目標。主題先行即指著手寫作前先要構造好假設樹,然後收集
資料和證據去驗證假設的真偽。假設的表述應落實到變數層次,賦予
操作性的定義,形成操作假設。


\subsubsection{示例:英文引用}
$\backslash$citet\{misc43\} 括号只有年份的引用方式,\citet{misc43}\label{misc43}

$\backslash$citep\{inproceedings\} 括号中有作者与年份的引用方式\citep{inproceedings}

\subsubsection{示例:中文引用}
$\backslash$citet\{inbook\} 括号只有年份的引用方式,\citet{inbook}\label{inbook}

$\backslash$citep\{booklet\} 括号中有作者与年份的引用方式\citep{booklet}


\clearpage
\subsubsection{示例:引用文獻的位置與格式}
\par 本大學碩士、博士論文引用文獻均統一使用著者-出版年制,各篇文獻
的標注內容由著者姓氏與出版年構成,並置於“( )”內。倘若只標
注著者姓氏無法識別該人名時,可標注著者姓名。多次引用同一著者
的同一文獻,在正文中標注著者與出版年,並在“( )”外以角標的
形式著錄引文頁碼,例如:(作者,年份)
頁碼 。在正文中引用同一著者
在同一年出版的多篇文獻時,出版年後應用小寫字母 a,b,c,…區別。
在正文中引用多著者文獻時,對歐美著者只需標注第一個著者的姓,
其後附“et al”;中國著者應標注第一著者的姓名,其後附“等”字,姓
氏與“等”之間留半字元空隙。

第二次引用時右上角標注上頁 \citet{f3}$^{\pageref{f3}}$
